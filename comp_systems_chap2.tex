\documentclass[12pt]{article}
\usepackage{amsmath}
\usepackage{amsfonts}

\begin{document}
\section{Conversion Functions}
\subsection{$B2U_w(\vec{x})$: Binary to Unsigned Integer}
\begin{eqnarray*}
B2U_w(\vec{x})&=&\displaystyle \sum \limits_{i=0}^{w-1} x_{i} 2^i\\
\end{eqnarray*}
Domain: Any binary number\\
Range: $\mathbb{Z}, \quad 0\leq y \leq 2^w-1$

\subsection{$B2T_w( \vec{x} )$: Binary to Signed}
\begin{eqnarray*}
B2T_w( \vec{x} ) = x_{w-1} ({-2}^{w-1}) + \displaystyle \sum \limits_{i=0}^{w-2} x_{i} 2^i
\end{eqnarray*}
Domain: Any binary number\\
Range: $\mathbb{Z}, \quad -2^{w-1}\leq y \leq 2^{w-1}-1$

\subsubsection{Binary conversion notes}
Both the above formulas are one-to-one both ways, that is any binary number represents one and only one unsigned or signed integer.

\subsection{Inverse Conversion Functions}
$U2B_w=\text{Inverse of } B2U$. Takes in a positive integer and converts it to binary. Use the typical method of converting a number to binary. If the number is negative use the method below.

$T2B_w=\text{Inverse of } B2T$. Takes in an integer and converts it to binary. Convert it to unsigned binary first, then take the two's complement.
\newpage
\section{Composite/Hard Conversion Functions}
\subsection{$U2T_w(\vec{x})$: Integer (Unsigned) to Integer (Signed)}
To do this conversion perform the following steps
\begin{enumerate}
	\item Convert the positive integer to binary
	\item Use $B2T_w(\vec{x})$ (formula 1.2) to convert the binary number to a signed integer.
\end{enumerate}

The formal equation is below
\begin{equation}
U2T_w(\vec{x})=B2T_w(U2B_x(\vec{x}))
\end{equation}
\[
 u(x) =
  \begin{cases}
   \exp{x} & \text{if } x \geq 0 \\
   1       & \text{if } x < 0
  \end{cases}
\]
\end{document}