\documentclass[12pt]{article}
\usepackage{amsmath}
\usepackage{amsfonts}
\usepackage{mathtools}

\begin{document}
\section{Conversion Functions}
\subsection{$B2U_w(\vec{x})$: Binary to Unsigned Integer}
\begin{eqnarray*}
B2U_w(\vec{x})&=&\displaystyle \sum \limits_{i=0}^{w-1} x_{i} 2^i\\
\end{eqnarray*}
Domain: Any binary number\\
Range: $\mathbb{Z}, \quad 0\leq y \leq 2^w-1$

\subsection{$B2T_w( \vec{x} )$: Binary to Signed}
\begin{eqnarray*}
B2T_w( \vec{x} ) = x_{w-1} ({-2}^{w-1}) + \displaystyle \sum \limits_{i=0}^{w-2} x_{i} 2^i
\end{eqnarray*}
Domain: Any binary number\\
Range: $\mathbb{Z}, \quad -2^{w-1}\leq y \leq 2^{w-1}-1$

\subsubsection{Binary conversion notes}
Both the above formulas are one-to-one both ways, that is any binary number represents one and only one unsigned or signed integer.

\subsection{Inverse Conversion Functions}
$U2B_w=\text{Inverse of } B2U$. Takes in a positive integer and converts it to binary. Use the typical method of converting a number to binary. If the number is negative use the method below.

$T2B_w=\text{Inverse of } B2T$. Takes in an integer and converts it to binary. Convert it to unsigned binary first, then take the two's complement.
\newpage
\section{Composite/Hard Conversion Functions}
\subsection{$U2T_w(x)$: Integer (Unsigned) to Integer (Signed)}
To do this conversion perform the following steps
\begin{enumerate}
	\item Convert the positive integer to binary
	\item Use $B2T_w(\vec{x})$ (formula 1.2) to convert the binary number to a signed integer.
\end{enumerate}

The formal equation is below
\begin{equation}
U2T_w(x)=B2T_w(U2B_x(x))
\end{equation}

\section{Truncating a binary number $k$ bits}
Takes in ANY binary number and truncates it $k$ bits. To perform the conversion use these steps:
\begin{enumerate}
\item Convert the number from binary to an unsigned integer. 
\item Letting that integer be $y$, perform $y \bmod 2^k$ to get another integer. If you want the unsigned form your done.
\item Letting that integer be $n$, perform $U2T_w(n)$
\end{enumerate}
This is useful for figuring out the result of
\begin{verse}
int y = 255;
char x = (char) y;
\end{verse}

\section{Binary Arithmetic From Integer Form}
This section shows how figure out what the addition/subtraction of two binary numbers is. It assumes that the two binary numbers are first converted to either signed for unsigned integers $x$ and $y$.
\subsection{Unsigned Arithmetic}
\begin{equation}
x+_w^u y = \begin{cases}
x+y & \text{if } x + y < 2^w \\
x+y-2^w & \text{if } 2^w \leq x+y < 2^{w+1}
\end{cases}
\end{equation}
Let $s=x+_w^u y$. If $s<x$ or $s<y$ then overflow has occurred.

\subsubsection{Additive Inverse}
\begin{equation}
-_w^u x = \begin{cases}
x & \text{if } x=0\\
2^w-x & \text{if } x>0
\end{cases}
\end{equation}

\subsection{Signed Arithmetic}
Given $x$, $y$ s.t. $-2^{w-1} \leq x,y \leq 2^{w-1}-1$
\begin{align}
x+_w^t y&=U2T_w\left(T2U(x) +_w^u T2U_w(y)\right)\\
&=U2T_w\left[(x+y) \bmod 2^w\right]
\end{align}
\subsubsection{Checking for Overflow}
\begin{tabular}{|c|c|c|}
\hline
If & Then & Notes \\ \hline
$2^{w-1} \leq x+y$ & $x+y-2^w$ & Positive Overflow\\
$-2^{w-1} \leq x+y \leq 2^{w-1}-1$ & $x+y$ & Normal\\
$x+y < -2^{w-1}$ & $x+y+2^w$ & Negative Overflow\\ \hline

\end{tabular}

\end{document}